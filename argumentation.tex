\documentclass[12pt]{article}

\begin{document}

\title{Book review}

\author{}
\author{Marcello Di Bello, Bart Verheij}

\date{}

\maketitle


Walton, D. (2016). \emph{Argument Evaluation and Evidence}. Berlin: Springer.

\section{Introduction}

\noindent Walton's  `Argument Evaluation and Evidence' (2016) is an ambitious book. It explores the nature of explanation,
expert opinion, knowledge and evidence. Walton makes the case that
contemporary methods developed in argumentation theory can help us shed light on these difficult topics.
%This is a valuable contribution by a key player in the
%field of argumentation theory, and an interesting read. 
%Walton presents the reader with a wealth of insights, theories, questions and solutions which might well scare them off upon first reading. A more leisurely paced exposition would have helped. Sometimes, as they say, less is more.
This review summarizes the main themes of the book (Section~\ref{summary}) and offers some comments (Section~\ref{comments}).

\section{Summary}
\label{summary}
\noindent Chapter~1 introduces concepts familiar to those in argumentation
theory. An argument consists of a set of premises and a conclusion,
and the premises can support the conclusion (pro arguments) or attack
it (con arguments). Arguments can be convergent (i.e., different
premises support the same conclusion), divergent (i.e., the same
premise supports different conclusions) or serial (i.e., the conclusion
of an arguments functions as the premise of another argument).
One goal of argumentation theory is to develop methods for evaluating the arguments for and against tentative conclusions. To this end, in previous work,
Walton investigated different types of arguments, for example, those
based on witness testimony and those based on expert opinion. In this
book, Walton intends to clarify the nature of another type of
argument, the so-called inference to the best explanation (Chapters 2
and 3), and further develop his examination of arguments based on
expert opinion (Chapters 4, 5 and partly 6). He also sets out to
address more theoretical questions about the nature of knowledge
(Chapter~7) and evidence (Chapter~8).

Chapter~2 discusses inference to the best explanation. Walton refers to an explanation `as an account of some connected sequence of events or actions that helps to transfer understanding from one party to another through a process of communication' (p. 64). Sometimes such an explanatory account contains an anomaly, i.e.,---as made explicit in Chapter~3--- `something the explainee does not understand in an account, even though she understands the rest of the account' (p. 78). In such a situation, a process of inference to the best explanation is appropriate. Consider an
anomalous fact, for example, the sudden death of a person. Competing
explanations of this fact are possible, such as: the neighbor killed
the victim; a stranger did; the death was natural; the death was
accidental. An inference to the best explanation selects the
explanation that outperforms its rivals, where each explanation takes
the form of a story, i.e., a spatiotemporal sequence of events and actions.
Walton distinguishes a backward argumentation scheme (p. 65). The premises of the scheme are a set of data and a series of explanatory accounts, some more successful than others, and the scheme's conclusion is the most successful explanation. The scheme comes with critical questions such as: How satisfactory is each account as an explanation? How much better is the best explanation compared to the others? Should the dialogue be continued before drawing a conclusion?   
%Walton identifies a number of key questions for assessing each
%competing story qua explanation. 
%Is it logically consistent? Is it
%plausible? Does the story leave something unexplained? \textbf{I CANNOT FIND THESE QUESTIONS} 
%Yet, a story
%that is best in terms of consistency, plausibility and explanatory
%coverage is not necessarily the best explanation. The story, after
%all, could still be false and entirely made up. 
What is needed,
Walton argues, is a way to combine explanations with arguments in a dialogical system. This
is the problem addressed in the next chapter.

Chapter~3 develops a dialogue-based framework to assess stories qua
explanations. Suppose an interlocutor puts forward an explanation for
a known fact, while the other interlocutor challenges it and
formulates an alternative. The proposed explanation is successful and
better than the alternative, Walton argues, provided it survives
challenges. A potential problem here is subjectivity. How can we avoid
making the success of an explanation too much dependent on those who
challenge it? Walton's answer is that explanations must be supported
by arguments. So, the "explanation game" shifts to an 
"examination game". In the explanation game, the conclusion at issue
is the proposed explanation itself. The explanation should be
supported by arguments, and the competing explanations, if they are to
be excluded, should lack appropriate argumentative support.

Chapter~4 and 5 (and partly Chapter~6 on the distinction between
correlation and causation) assess arguments based on expert opinion.
A typical argument from
expert opinion has the form "Expert A says X. Therefore X". The motivation for the chapter is that experts
disagree and in many cases have been proven wrong. We should therefore not
trust them blindly, and argumentation theory offers us a method to
orient ourselves. These chapters contain interesting case studies. In Chapters~4 and 5 most examples are 
from art history, in which art critics and scientists disagreed about
the artwork's authenticity. Walton reconstructs the conflicting
arguments by dissecting, charting and weighing them through critical
questions, such as: How credible is the expert? Is the expert prepared
in the field? Is the expert biased? Chapter~6 discusses arguments concerning public health issues, such as southern Pacific weather patterns and flu pandemics. The examples are discussed in the context of 
intelligent systems that can support the assessment of
conflicting arguments and the weighing of pros and cons.

The two final chapters of the book are devoted to deep philosophical
questions about the nature of knowledge and its relation to evidence and arguments.

Chapter~7 defends a process-based, fallible account of knowledge. On
this account, knowledge is the result of a process of dialogical inquiry in which
propositions are tested and scrutinized in light of the evidence and arguments
available. If propositions survive testing and the supporting evidence is
strong enough to meet the applicable standard of proof, they become
knowledge. New evidence may later undermine propositions
previously known. Knowledge is thus both fallible and revocable.
Walton discusses the idea---common in epistemology---that knowledge implies
truth. Knowledge is than true belief with an extra property, such as proper justification.
Walton
's account departs from this idea of `kowledge as true-belief-plus' and replaces it with defeasible veracity, that is, knowledge of
P defeasibly implies the truth of P. 
This process-based, defeasible
account of knowledge is informed by the theories of Pierce and Popper,
and Walton complements these with recent developments in the formal and computational study of argumentation,
for example, Carneades, ASPIC+ and DefLog.
Some might object that items of basic knowledge such as "I have hands"
are not arrived at by means of a process. Walton responds that even "I
have hands" derives from a defeasible process of knowledge
acquisition, of the form "the senses tell me that I have hands unless
an exception holds (e.g., I am hallucinating), and since there is no
evidence that an exception holds, it can be concluded, defeasibly,
that I have hands". In this way, the proposition "I have hands" is not immediate and fits into a model of fallible, process-based
knowledge. In his discussion, Walton continues from his earlier work in which he discussed a pragmatic conception of knowledge in which our everyday knowledge as stored in our memory, our `knowledge base', is incomplete and fallible (p. 212).

Chapter~8, the last in the book, discusses the relationship between
arguments and evidence. Walton begins by noting an ambiguity in the
use of the word `evidence'. Broadly speaking, any argument that
supports a certain conclusion provides evidence for that conclusion. More
narrowly, only certain specific kinds of reasons count as evidence, for instance those based on observations statistics or other scientific results. He addresses the issue of distinguishing between argument and evidence, and at the end of the book discusses three factors (p. 276). First it matters whether the kinds of evidence used are right for the argumentation in a given case. Second the argument given should fit a recognized argumentation scheme. And third the argument should be `found in the knowledge base representing the evidential findings in the case that have been accepted as factual'.

\section{Comments}
\label{comments}

\noindent As in much of his work, Walton uses his well-developed perspective on argumentation and dialogue to investigate hard problems also investigated in other fields of investigation. 
Here he uses his extensive experience in applying the tools of argumentation theory---many of which he developed or has helped develop---to shed light on the complex themes of the
relation between arguments and explanations; arguments and knowledge;
and arguments and evidence. . 
Along the way he shows his extensive interests and grasp of relevant work, aiming to integrate scholarship from neighboring fields such as epistemology and philosophy of science, and to include relatively new developments such as the formal and computational study of argumentation (associated with the biennial COMMA conference series and the journal `Argument and Computation). In these ways, Walton's book is a useful and interesting scholarly contribution.

As expected for a book on these hard themes, there is room for further exploration. A first area of further research is the development of a theory of (the
best) explanation. In Walton's book, not much is said about the notion of
explanation itself. A key part of the notion is that an
explanation is meant to address an anomalous fact or that it should
convey to an interlocutor an understanding of the fact to be
explained. But it seems that there are explanations of facts that are not thought of as anomalous. We seek, for example, an
explanation of why the sun rises every morning, which is hardly an
anomalous fact, or if it is, it would be
anomalous in a different, more specific way to be made explicit. Further, while explaining might sometimes
involve the act of conveying to an interlocutor a certain
understanding of the fact to be explained, this is not always the
case. Relativity theory, for example, explains a number of things in a
way that most people do not understand. Presumably Walton's focus is mostly on a
communicative theory of explanation or a common sense theory. It would be interesting to know more about the scope of the
intuitions about explanation made explicit by Walton.

A second area of further work is how Walton's discussions relate to other work in philosophy, especially epistemology and philosophy of science. In particular, Walton begins by attacking the thesis that knowledge of P implies
the truth P, and goes on to offer a fallibilist theory of
knowledge. Many  epistemologists in the analytic tradition have taken a similar perspective. On the one hand, it is true that most analytic
epistemologists believe that knowledge implies truth. But, at the same
time, most of them also believe that knowledge is fallible (contrary
to what Walton seems to think). Many contemporary theories of
knowledge in the analytic tradition, in fact, are fallibilist theories
but also embrace the thesis that knowledge implies truth. THIS SEEMS PARADOXICAL BUT <SOME SPECIFICS>. See, for
example, the survey piece on the Stanford Encyclopedia of Philosophy
"The Analysis of Knowledge". So, pace Walton, there need not be a
conflict between the fallibility of knowledge and the thesis that
knowledge implies truth. 
IS THE ABOVE TRUE? I SEE TRUE KNOWLEDGE AND FALLIBLE SENSES. THIS PARAGRAPH NEEDS SOME WORK.

Third, consider how Walton makes an argument of dubious cogency. On
page 232, he mentions a well-known skeptical challenge. If I see a red
lamp, how do I know that I am not hallucinating? Walton suggests that
the lack of evidence that I am hallucinating supports the conclusion
that I am not hallucinating, because if I were, there would be
evidence that I was. So, Walton argues, given that there is no such
evidence, it can be concluded, defeasibly, that the lamp is in fact
red. This reasoning will appear preposterous to many epistemologists.
The fact that I am hallucinating makes it impossible to recover any
evidence that I am or am not hallucinating. The fact that there is no
evidence that I am hallucinating is precisely what I would expect if I
were hallucinating. This is why skepticism is so hard to dismiss.
Arguably, Walton did not aim to answer the skeptical challenge, but
rather, simply make explicit an ordinary (though somewhat dubious)
pattern of reasoning which could be applied in situations that do not
involve radical skepticism. It is plausible to say that absence of
evidence is evidence of absence, at least in some scenarios. Still,
Walton should have told us what such scenarios are.
I DON'T REALLY UNDERSTAND THE POINT OF THIS PARAGRAPH. MY IMPRESSION IS THAT THIS IS ABOUT TWO DISTINCT THEMES: what counts as knowledge for a knower (perhaps this is what you would call justified belief, justified-for-the-knower) and what count as knowledge for an evaluator of that knower. A hallucinating knower could then be justified in believing something, but an outside evaluator would understand that that is a false justified belief; hence no knowledge. END OF REMARK

Finally, consider how Walton fails to address a crucial
epistemological problem. In Chapter~8, he gives a theory of evidence
and how evidence is related to arguments. But he does not face the
elephant in the room. This is the question, how can an infinite
regress of justification be avoided? I THINK HE DOES BY USING A KNOWLEDGE BASE (INTERNAL TO THE KNOWER) AS A BASIS. For one, he suggests that
evidence consists of propositions that cannot be further questioned,
but also admits that such evidential propositions, in certain
circumstances, can be questioned. But Walton does not specify what
triggers the need of further questioning, nor when the further
questioning is legitimate or inappropriate. Does the chain of
justification stop at some point or does it regress back ad infinitum?
Perhaps, this is not the sort of question we should be asking, but if
not, Walton should tell us why.
WHAT REMAINS OF THIS CHALLENGE AFTER THE KNOLWEDGE BASE REMARK? REPHRASE? REPHRASE AS A SUGGESTION HOW WALTON"'S WORK RELATES TO A KEY QUESTION IN EPIST?

There is a specific contribution Walton makes that distinguishes it especially from analytic philosophy. It is the process-based, or
inquiry-based, approach to a theory of knowledge. Analytic
philosophers seem to have been mostly concerned with the statics of knowledge
-- that is, with identifying conditions under which a certain
evidential state, held by a group or by an individual, counts as
knowledge. BUT WHAT ABOUT DYNAMIC EPISTEMIC LOGIC? en.wikipedia.org/wiki/Dynamic\_epistemic\_logic END OF REMARK Despite some recent work, analytic philosophers have not
been much concerned with the dynamics of knowledge -- that is, with
the process by which knowledge is acquired and lost. The process-based
perspective which Walton brings to the table is certainly interesting
from a philosophical point of view.

\section{Concluding remarks}

\noindent All in all, this book offers a wealth of
insights, ideas and interesting examples. It is written by one of the
foremost experts in the field, and those interested in argumentation
theory and its epistemological underpinnings will certainly profit a
great deal from reading it.

\end{document}
