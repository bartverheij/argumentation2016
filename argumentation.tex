\documentclass[12pt]{article}

\begin{document}

\title{Book review}

\author{}
\author{Marcello Di Bello, Bart Verheij}

\date{}

\maketitle


Walton, D. (2016). \emph{Argument Evaluation and Evidence}. Berlin: Springer.

\section{Introduction}

\noindent Walton's \emph{Argument Evaluation and Evidence} (2016) is an ambitious book. It explores the nature of explanation,
expert opinion, knowledge and evidence. Walton makes the case that
contemporary methods developed in argumentation theory can help us shed light on these difficult topics.
%This is a valuable contribution by a key player in the
%field of argumentation theory, and an interesting read. 
%Walton presents the reader with a wealth of insights, theories, questions and solutions which might well scare 
%them off upon first reading. A more leisurely paced exposition would have helped. Sometimes, as they say, less is more.
This review summarizes the main themes of the book (Section~\ref{summary}) and offers some comments, 
mostly on the relationship between argumentation theory and contemporary analytic epistemology (Section~\ref{comments}).

\section{Summary}
\label{summary}
\noindent Chapter~1 introduces basic concepts from argumentation
theory. An argument consists of a set of premises and a conclusion,
where the premises can support the conclusion (pro arguments) or attack
it (con arguments). Arguments can be convergent (i.e.\ different
premises support the same conclusion), divergent (i.e.\ the same
premise supports different conclusions) or serial (i.e.\ the conclusion
of an arguments functions as the premise of another argument).
One goal of argumentation theory is to develop methods for evaluating 
arguments for and against tentative conclusions. To this end, in previous work,
Walton investigated different types of arguments, for example, those
based on witness testimony and those based on expert opinion. In this
book, Walton intends to clarify the nature of another type of
argument, the so-called inference to the best explanation (Chapters 2
and 3), and further develop his examination of arguments based on
expert opinion (Chapters 4, 5 and  6). He also sets out to
address epistemological questions about the nature of knowledge
(Chapter~7) and evidence (Chapter~8).

Chapter~2 discusses inference to the best explanation. Walton refers to an explanation 
`as an account of some connected sequence of events or actions that helps to transfer understanding 
from one party to another through a process of communication' (p.\ 64). 
Such an account is typically aimed to explain an anomaly, defined as 
%As made explicit in Chapter~3, an anomaly is 
`something the explainee does not understand' % in an account, 
%even though she understands the rest of the account'
(p.\ 78). 
%In such a situation, the need for an explanation arises. 
%a process of inference to the best explanation is appropriate. 
To illustrate, consider an
anomalous fact, for example, the sudden death of a person. This event can be explained in a number of ways: 
the death occurred naturally; 
the neighbor killed the victim; a stranger did; etc. An inference to the best explanation selects the
most successful explanation which outperforms its rivals, where each explanation takes
the form of a story, that is, a spatiotemporal sequence of events and actions.
To model inference to the best explanation, Walton proposes 
an argumentation scheme (p.\ 65). The premises of the scheme are threefold: (1) 
a set of data and facts to be explained; (2) statements that each of the competing explanations explains 
the data and facts; (3) the statement that one of the explanations is the most successful. 
%and a series of explanatory accounts, some more successful than others, and 
The scheme's conclusion is the most successful explanation. The scheme comes with critical questions, such as: 
How satisfactory is each explanation? How much better is the best explanation compared to the others? 
%Should the dialogue be continued before drawing a conclusion?   
%Walton identifies a number of key questions for assessing each
%competing story qua explanation. 
%Is it logically consistent? Is it
%plausible? Does the story leave something unexplained? \textbf{I CANNOT FIND THESE QUESTIONS} 
%Yet, a story
%that is best in terms of consistency, plausibility and explanatory
%coverage is not necessarily the best explanation. The story, after
%all, could still be false and entirely made up. 
This %backward 
argumentation scheme, however, still leaves the process of selecting the best explanation somewhat opaque. 
What is needed, Walton argues, is a way to combine explanations with supporting 
reasons and arguments in a dialogical system. This
is the topic of the next chapter.

Chapter~3 develops a dialogue-based framework to assess stories \textit{qua}
explanations. Suppose an interlocutor puts forward an explanation for
a known fact, while the other interlocutor challenges it and
formulates an alternative. The proposed explanation is successful and
better than the alternative, Walton argues, provided it can survive all
the challenges that the other interlocutor poses. A potential problem here is subjectivity. How can we avoid
making the success of an explanation too dependent on the interlocutor's challenges? Walton's answer is that explanations must be supported
by arguments, %So, the ``explanation game" shifts to an 
%``argumentation game". In the explanation game, the conclusion at issue
%is the proposed explanation itself. In the argumentation game, the explanation should be
%supported by arguments, 
and the competing explanations, if they are to
be excluded, should lack appropriate argumentative support. 
The best explanation, then, will be the one that is 
supported by the strongest argument.

Chapter~4, 5 and 6 
%on the distinction between
%correlation and causation) 
assess arguments based on expert opinion.
A typical argument from
expert opinion has the form ``Expert A says X; therefore X". The motivation for these chapters is that experts
disagree and in many cases have been proven wrong. We should therefore not
trust them blindly, and argumentation theory offers us a method to
orient ourselves in the face of conflicting expert testimonies. These chapters contain 
interesting case studies. In Chapters~4 and 5, the examples are 
from art history, in which art critics and scientists disagreed about
the authenticity of artworks. Walton reconstructs the conflicting
arguments by dissecting, charting and weighing them through critical
questions, such as: How credible is the expert? Is the expert prepared
in the field? Is the expert biased? Chapter~6 is about the distinction between correlation and causation. 
It discusses arguments about public health issues, such as southern Pacific weather patterns and flu pandemics. The examples are discussed in the context of 
intelligent systems that can support the assessment of
conflicting arguments and the weighing of reasons pros and cons.

The two final chapters of the book are devoted to deep philosophical
questions about the nature of knowledge and its relation to evidence and arguments.

Chapter~7 defends a process-based, fallible account of knowledge. On
this account, knowledge is the result of a process of dialogical inquiry in which
propositions are tested and scrutinized in light of the evidence and arguments
available. If a proposition survives testing and the supporting evidence is
strong enough to meet the applicable standard of proof, it becomes an item of
knowledge. On Walton's account, new evidence that contradicts existsing evidence 
may later defeat propositions previously known. In this sense, 
knowledge is fallible, never definitive and subject to change. 
%(because unassailable certainty is not required) %and revocable (
%because it could be later defeated.
%Walton criticizes the idea---common in analytic epistemology---that knowledge implies
%truth. %Knowledge is than true belief with an extra property, such as proper justification.Walton
%'s account departs from this idea of `kwoledge as true-belief-plus' and 
%Instead of claiming that knowing a proposition implies that the proposition must be true, 
%Walton argues that knowing a proposition defeasibly implies the truth of the proposition.
%This shiftreplaces it with defeasible veracity, that is, . 
This process-based, defeasible
account of knowledge is informed by the theories of Pierce and Popper,
but Walton complements these with recent developments in the formal and computational study of argumentation,
for example, Carneades, ASPIC+ and DefLog. REFERENCES
Some might object that items of basic knowledge such as ``I have hands"
are not arrived at by means of a process. Walton responds that even ``I
have hands" derives from a defeasible process of knowledge
acquisition. Such a process can be roughly described as follows: 
since the senses attest that I have hands 
and since there is no evidence to the contrary (e.g.\ I am hallucinating), it can be concluded, defeasibly,
that I have hands. In this way, the proposition ``I have hands" is not immediate and fits into a model of fallible, process-based
knowledge. In his discussion, Walton develops his earlier work in which he proposed a pragmatic 
conception of knowledge. REFERENCES On this conception, 
everyday knowledge is stored in our memory---what computer scientists 
call the knowledge base---and this knowledge is both incomplete 
and fallible (p.\ 212).

Chapter~8, the last in the book, discusses the relationship between
arguments and evidence. Walton begins by noting an ambiguity in the
use of the word `evidence'. Broadly speaking, any argument that
supports a certain conclusion provides evidence for that conclusion. More
narrowly, only certain specific kinds of reasons count as evidence, for instance those based on observations, statistics or other scientific results. He addresses the issue of distinguishing between arguments and evidence, and at the end of the book discusses three factors. First, it matters whether the kinds of evidence used are right for the argumentation in a given case. Second, the argument given should fit a recognized argumentation scheme. 
And third, the argument should be `found in the knowledge base representing the evidential findings in the case that have been accepted as factual'
(p.\ 276).

\section{Comments}
\label{comments}

\noindent %As in much of his work, Walton uses his well developed perspective on argumentation and dialogue to investigate hard problems 
%that have also been investigated by other disciplines, especially analytic philosophy and epistemology. 
%REPLACED tackled (there are no definite answers in these disciplines)
%Here he 
In this book, Walton applies the tools of argumentation theory---many of which he developed or has helped develop---to shed light on 
the relation between arguments and explanations; arguments and knowledge;
and arguments and evidence. 
%Along the way he shows his extensive interests and grasp of relevant work, aiming to 
He integrates scholarship from neighboring fields such as epistemology and philosophy of science, and includes 
new developments such as the formal and computational study of argumentation (associated with the biennial COMMA conference series 
and the journal `Argument and Computation). In these ways, Walton's book is a useful and interesting scholarly contribution.

As expected for a book on these difficult topics, there is room for further exploration. A first area of further research is the development of a theory of (the
best) explanation. In Walton's book, not much is said about the notion of
explanation itself. A key part of the notion is that an
explanation is meant to address an anomalous fact or that it should
convey to an interlocutor an understanding of the fact to be
explained. But it seems that there are explanations of facts that are not thought of as anomalous. We seek, for example, an
explanation of why the sun rises every morning, which is hardly an
anomalous fact, or if it is, it would be
anomalous in a different, more specific way to be made explicit. Further, while explaining might sometimes
involve the act of conveying to an interlocutor a certain
understanding of the fact to be explained, this is not always the
case. Relativity theory, for example, explains a number of things in a
way that most people do not understand. Presumably, Walton's focus is mostly on a
communicative theory of explanation or a common sense theory. A broader theory of explanation 
could provide more insight about the scope of the intuitions made explicit by Walton.

A second area of further work is how Walton's theory of knowledge relates to existing theories 
in contemporary analytic epistemology. 
%A couple of examples should clarify this point. 
Many epistemologists hold that a proposition is known 
only if it is well supported by the evidence. This can hardly be questioned, 
and Walton would certainly agree.  
But here is where Walton distances himself 
from analytic epistemology, in a way that is fully motivated. 
While many epistemologists hold that 
for a proposition to be known, it must be true, Walton thinks that the truth requirement 
must be abandoned. In Chapter~7, he argues that philosophers have not sufficiently motivated why knowledge 
should require truth. This is correct. The truth requirement, in fact, is almost 
entirely taken for granted in contemporary epistemology. But Walton offers 
(what he takes to be) a stronger objection: the fallibility of knowledge is incompatible with 
the thesis that knowledge implies truth.  
In other words, if knowledge is fallible, 
it cannot imply truth. At first blush, this seems correct. If knowledge implies truth, it admits of no mistake, because whenever 
something is known, it must be true. 
Hence, it would seem, if knowledge implies truth, knowledge must be infallible. 
%However, the mainstream position among analytic epistemologists is different. 
And yet, many contemporary theories of knowledge in the analytic tradition are fallibilist 
but also embrace the thesis that knowledge implies truth. 
REFERENCES
%See, for example, the survey piece on the Stanford Encyclopedia of Philosophy ``The Analysis of Knowledge". 
% REMOVED as it does not seem to directly provide the relevant insights
How can that be? 
Consider an example. 
An eyewitness claims she saw the defendant near the crime scene at a particular time.
Suppose the testimony is scrutinized and checked against other eyewitness reports, and nothing 
wrong is found with the testimony.
Further, the claim made by the witness is true, that is, the defendant was in fact at the crime scene 
at the time the witness claims he was. Since the claim is both well supported by the evidence and true,  
it counts as an item of knowledge (under the theory that knowledge implies truth). 
Can this still be fallible knowledge? It can. 
Suppose new evidence---\textit{prima facie} reliable evidence---comes up, and this evidence 
contradicts the eyewitness testimony. The claim that the defendant was near the crime scene would still 
be true, but in light of the new evidence, it would not longer be well supported by the evidence. The 
claim would therefore not be an item of knowledge. 
%Even if it is thought to imply the truth of what is known, 
%knowledge is fallible in the sense that it can 
%be defeated by further evidence. 
So, \textit{pace} Walton, the fallibility of knowledge is compatible with 
the thesis that knowledge implies truth. Does Walton have any stronger reason 
to reject the requirement that knowledge implies truth?






%The fallibility of knowledge means that one's knowledge of a proposition 
%does not require having ruled out every possibility of error or deception relative to the proposition 
%being known. For example, I can know, 
%presumably, that I have hands, even without ruling out every far fetched possibility of error, such as, being systematically deceived, being a brain-in-a-vat, being always hallucinating. 
%This sense of fallibility is compatible with the thesis that knowledge implies truth. If one knows $p$, then $p$ must be true, but this does not require one to possess evidence so powerful to exclude every possibility of error. 

%I AM NOT YET HAPPY WITH THIS PART. For two reasons: 1) It uses anti-common sense examples (brains in vats). 2) It is not entirely transparent. I tried to start thus, but encountered an incomplete grasp of the subject matter:

%In a fallibilist, truth-implying theory of knowledge, four situations must be distinguished:

%\begin{enumerate}
%	\item A belief is justified (given current evidence) and true.
%	\item A belief is justified (given current evidence) and false.
%	\item A belief is unjustified (given current evidence) and true.
%	\item A belief is unjustified (given current evidence) and false.
%\end{enumerate}

%\noindent Only a belief in the first of these four situations counts as knowledge. HERE IS WHERE I GOT STUCK: AS THIS MAKES KNOWLEDGE INFALLIBLE. ONLY WHEN TYPE 2 ALSO COUNTS AS KNOWLEDGE WE CAN HAVE FALLIBLE KNOWLEDGE. DOES IT?

 %ONLY WHEN TYPE 2 ABOVE COUNTS AS KNOWLEDGE. BUT THEN KNOWLEDGE IS NOT JUSTIFIED TRUE BELIEF AND I GET CONFUSED. (NEED TO CHECK POLLOCK'S BOOK ON EPISTEMOLOGY. NOT WITHIN REACH ;-))
%It would be interesting to know how Walton's account of knowledge could be appropriately modified to accommodate this other sense of fallibility.

 %Many  epistemologists in the analytic tradition have taken a similar perspective. On the one hand, it is true that most analytic
%epistemologists believe that knowledge implies truth. But, at the same
%time, most of them also believe that knowledge is fallible (contraryto what Walton seems to think). 
%IS THE ABOVE TRUE? I SEE TRUE KNOWLEDGE AND FALLIBLE SENSES. THIS PARAGRAPH NEEDS SOME WORK.


Third, in this book Walton touches upon the 
perennial problem of skepticism, but does so in a way that requires 
more explaining. Here is a classical example. 
I see a cat in from of me. The sensory evidence available to me supports 
the conclusion that there is indeed a cat in front of me. 
Can I thereby conclude that I know there is a cat in front of me?
After all, I could be hallucinating, seeing a hologram, or anything of 
that sort which would undermine my presumptive knowledge. This is the skeptical challenge.
%he mentions a well-known skeptical challenge. If I see a red
%lamp, how do I know that I am not hallucinating? 
Walton suggests (p.\ 232) that the lack of evidence that I am hallucinating supports the conclusion
that I am not hallucinating, because if I were hallucinating, there would be
evidence that I was. So, Walton argues, given that there is no such
evidence, it can be concluded, defeasibly, that the 
there is a cat in fort of me.  
%For one, this reasoning seems perfectly good. 
%We use it all the time. We draw conclusions even though unexpected events, circumstances or exceptions 
%might make such conclusions false. For example, we believe that the next 
%bus will arrive on time (\textit{unless} there is evidence to the contrary). 
%Insofar as we do not have any evidence to the contrary, we are content 
%to conclude that next bus will in fact arrive on time. It is plausible to say, then, that absence of
%evidence is evidence of absence, at least in some scenarios.
%But on what basis is this reasoning pattern justified?
%Walton seems to take it for granted without offering any clear justification, except---perhaps---that 
%we use it often. 
%Consider again the case of hallucinating and suppose I am in fact hallucinating. 
But there are reasons to pause here. If I were hallucinating, this would make 
it impossible to recover any evidence that I am or am not hallucinating. The fact that there is no
evidence that I am hallucinating is precisely what I would expect if I
were hallucinating. The same point can be put in terms of an explanation. 
Both the hypothesis that I am hallucinating and the hypothesis that I am not hallucinating 
equally explain (or predict) the absence of evidence that I am hallucinating.
So, neither of the hypotheses can be preferred, and that is why skepticism is so hard to dismiss.
The reasoning pattern according to which the absence of evidence that I am hallucinating 
licenses---even defeasibly---the conclusion that I am not hallucinating (and, in turn, licenses the conclusion that 
there is a cat in front of me because I see one) seems dubious. %Arguably, this is the conclusion that, pragmatically, makes more sense. 
On what basis is this reasoning pattern justified?
Walton seems to take it for granted without 
offering any clear justification. %, except---perhaps---that 
%we use it often.
%Why does Walton say that if there is no evidence that I am hallucinating 
%I can conclude, defeasibly, that I am not hallucinating? 
%Chapter~7 could use a more careful treatment of this theme.


%So, in the hallucination case it makes little sense to say that since there is no 
%evidence that I am hallucinating, I can assume that I am not. 
%It would be interesting to know to what extent Walton aims to address epistemological question, or rather, 
%operationalized certain common sense inferences that e make without justifying them at a deeper level. 
%Arguably, Walton did not aim to answer the skeptical challenge, but
%rather, simply make explicit an ordinary (though somewhat dubious)
%pattern of reasoning which could be applied in situations that do not
%involve radical skepticism.  Still,
%Walton should have told us what such scenarios are.
%I DON'T REALLY UNDERSTAND THE POINT OF THIS PARAGRAPH. MY IMPRESSION IS THAT THIS IS ABOUT TWO DISTINCT THEMES: what counts as knowledge for a knower (perhaps this is what you would call justified belief, justified-for-the-knower) and what count as knowledge for an evaluator of that knower. A hallucinating knower could then be justified in believing something, but an outside evaluator would understand that that is a false justified belief; hence no knowledge. END OF REMARK

Fourth, it would be interesting to know where Walton stands on the foundationalism v.\ coherentism 
debate in epistemology. For instance, Pollock has contributed to this debate using an argumentation perspective. REFERENCES. Foundationalists believe that knowledge must rest on certain basic propositions, which cannot be further questioned. By contrast, coherentists believe that knowledge emerges from a web of beliefs, so that 
the combination of mutually reinforcing beliefs constitutes 
the edifice of knowledge. In Chapter~8, %Walton gives a theory of evidence and how evidence is related to arguments. Her seems to 
Walton seems to lean toward foundationalism by postulating that there exists a knowledge base, internal to each knower or group of knowers. 
Propositions that belong to the knowledge base are not further questioned.
This resembles foundationalism. But if so, the question remains of how 
the knowledge base is constructed. Could any proposition count as 
part of the knowledge base? Are there criteria for a proposition 
to be part of a knowledge base? Is the choice pragmatically determined 
by the needs of the knowers, or are there more objective, 
or intersubjective, criteria that apply? Is the knowledge base subject to change?
%THE PRECEDING PART I SLIGHTLY EDITED

%consider how Walton fails to address a crucial
%epistemological problem. I THINK HE DOES BY USING A KNOWLEDGE BASE (INTERNAL TO THE KNOWER) AS A BASIS. For one, he suggests that
%evidence consists of propositions that cannot be further questioned,
%but also admits that such evidential propositions, in certain
%circumstances, can be questioned. But Walton does not specify what
%triggers the need of further questioning, nor when the further
%questioning is legitimate or inappropriate. Does the chain of
%justification stop at some point or does it regress back ad infinitum?
%Perhaps, this is not the sort of question we should be asking, but if
%not, Walton should tell us why.
%WHAT REMAINS OF THIS CHALLENGE AFTER THE KNOLWEDGE BASE REMARK? REPHRASE? REPHRASE AS A SUGGESTION HOW WALTON"'S WORK RELATES TO A KEY QUESTION IN EPIST?

Let us conclude by mentioning 
a couple of original contributions Walton makes 
which should be of interest to those 
in mainstream analytic epistemology. The first is a process-based, or
inquiry-based, approach to a theory of knowledge. Analytic
epistemologists have been mostly concerned with the statics of knowledge, 
that is, with identifying conditions under which a certain
evidential state, held by a group or by an individual, counts as
knowledge. Despite some recent work (see, for example, 
Dynamic Epistemology Logic), most analytic philosophers have not
been much concerned with the dynamics of knowledge, that is, with
the process by which knowledge is acquired and lost. Such a process-based
perspective as Walton brings to the table is particularly interesting
from a philosophical point of view. 

Another interesting contribution is Walton's dialogical and arguments-based 
approach to epistemology. This is not a new idea in philosophy. The suggestion 
that knowledge has to do with answering challenges can be found, among others,
in the writings of John Austin and more recently Nicholas Rescher. 
But philosophers are often concerned with general 
theories. Walton offers more details and 
brings his distinctive perspective.
Walton uses argument schemes and critical questions
to describe the structure of justification and evidence. 
Arguments can be dialogically tested, strengthen or undermined, 
in a variety of ways, for example, by challenging their premises 
or the connection between premises and conclusion.  
%This structure consists of arguments, which have premises 
%conclusions, but also rests of tacit assumptions. 


\section{Concluding remarks}

\noindent All in all, this book offers a wealth of
insights, ideas and interesting examples. It is written by one of the
foremost experts in the field, and those interested in argumentation
theory and its epistemological underpinnings will profit 
by reading it.

\end{document}
